\documentclass[]{article}  % list options between brackets
\usepackage{}              % list packages between braces

% type user-defined commands here

\begin{document}

\title{\TeX\ and \LaTeX}   % type title between braces
\author{Tom Scavo}         % type author(s) between braces
\date{October 27, 1995}    % type date between braces
\maketitle

\begin{abstract}
  A brief introduction to \TeX\ and \LaTeX
\end{abstract}

\section{\TeX}             % section 1
\subsection{History}       % subsection 1.1

Bij de computationele benadering vertrekt men van de volgende vragen:
1)Welke informatie is uberhaupt beschikbaar in een beeld?
Bv: .............
2)Welke informatie hebben we nodig voor het oplossen van een bepaald perceptueel probleem?
Bv: .....
3)Hoe kan men (2) afleiden uit (1)

Meestal komt men tot de vaststelling dat er inderdaad veel informatie aanwezig is in het beeld (zoals de ecologische benadering steeds beweerd) – maar slechts impliciet. Men zal die informatie trapsgewijs moeten expliceren. Er zijn maw. Meerdere tussenstappen nodig om van de input naar de output te geraken (zoals de cognitieve benadering ook steeds beweert; De computationele benadering is dus een soort van synthese tussen beide benaderingen. 


\section{\LaTeX}           % section 2
\subsection{Usage}         % subsection 2.1

Bij de computationele benadering vertrekt men van de volgende vragen:
1)Welke informatie is uberhaupt beschikbaar in een beeld?
Bv: .............
2)Welke informatie hebben we nodig voor het oplossen van een bepaald perceptueel probleem?
Bv: .....
3)Hoe kan men (2) afleiden uit (1)

Meestal komt men tot de vaststelling dat er inderdaad veel informatie aanwezig is in het beeld (zoals de ecologische benadering steeds beweerd) – maar slechts impliciet. Men zal die informatie trapsgewijs moeten expliceren. Er zijn maw. Meerdere tussenstappen nodig om van de input naar de output te geraken (zoals de cognitieve benadering ook steeds beweert; De computationele benadering is dus een soort van synthese tussen beide benaderingen. 


\begin{thebibliography}{9}
  % type bibliography here
\end{thebibliography}

\end{document}

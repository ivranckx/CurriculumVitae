\documentclass[]{article}

%opening
\title{}
\author{}

\begin{document}

\maketitle

\begin{abstract}

\end{abstract}

\section{}

Betreft: sollicitatie voor lector Vroedkunde.

Ten aanzien van Mevrouw Kim Van Hoof, Opleidingshoofd Vroedkunde.

Geachte Mevrouw,
Beste Kim,

Na veel overwegingen — een motivatiebrief highlight immers meestal de persoonlijke competenties — heb ik besloten de vroedkundige centraal te stellen in mijn onderstaande betoog.

Vroedvrouwen, maar eveneens ook kersverse ouders of ervaren verpleegkundigen, worden in de praktijk frequent geconfronteerd met dikwijls onverwachte traumatische gebeurtenissen. Een postnatale depressie zoals onlangs weer in het nieuws verscheen vormt hier spijtig genoeg een courant terugkerende voorbeeld van.
Deze traumatische gebeurtenissen gaat ook frequent gepaard met een impact op het psychologisch welzijn van de vroedvrouw in kwestie. Het is daarom van uiterst, doorgedreven belang dat er actief onderzoek en ontwikkeling gebeurt naar methodologieën om deze traumatische gebeurtenissen te verwerken.
Omdat het psychisch welbevinden (mijnsinziens) dermate belangrijk is (ook in het dagdagelijks leven van elke moeder) zou ik hier graag rond willen werken binnen het kader van jullie expertisecentrum en de geassocieerde opleiding vroedkunde. De erkenning voor de urgentie én het belang van deze problematiek vormt een rode draad in mijn studiekeuze, mijn thesisonderwerp en verdere ontplooiing, gecombineerd met relevante werkervaring nadien.
Het is naar mijn meting van groot belang dat de studenten correct kunnen omgaan met de moderne technologie, in het bezit zijn van een goede wetenschappelijke scholing en voldoende expertise aan boord hebben om de dagdagelijkse vroedkundige uitdagingen adequaat te kunnen uitvoeren.
Een positieve attitude, een professionele opleiding en de juiste, ruimdenkende mentaliteit hierin belangrijk.

De reden waarom ik (vanuit zowel een intellectueel als empathisch perspectief) geopteerd heb om deze nieuwe uitdaging aan te gaan is omdat ik op deze manier belangrijke informatie wil overdragen, naast de wetenschap dat mijn advies het verschil kan maken op cruciale momenten.

Ik hoop dat mijn summiere geschetste motivatie u geïnspireerd heeft om mijn sollicitatie een vervolg te geven.

Met Vriendelijke groeten,
Tinne Wouters


\end{document}

%----------------------------------------------------------------------------------------
%	SECTION TITLE
%----------------------------------------------------------------------------------------

\cvsection{Werkervaring}

%----------------------------------------------------------------------------------------
%	SECTION CONTENT
%----------------------------------------------------------------------------------------

\begin{cventries}

%------------------------------------------------

\cventry
{Nachtverpleegkundige} % Job title
{Woonzorgcentrum Edelweiss} % Organization
{Liersesteenweg 165/171, 3130 Begijnendijk.} % Location
{2019 - heden} % Date(s)
{ % Description(s) of tasks/responsibilities
\begin{cvitems}
	\item Verpleegkundige.
\end{cvitems}
}
%	\begin{cvitems}
%		\item Docent van het wetenschappelijk vak "Evidentie en Onderzoek", 120-tal studenten.
%		\item Begeleider van BANABA studenten pediatrie bij hun bachelorproef.
%		\item Stagebegeleiding studenten verpleegkunde.
%		\item Evalueren van leerprocessen, bijsturen en actief werken rond de verdere uitbouw van de opleiding (stuurgroepen).
%		\item Geven van vaardigheidtrainingen (skillslab) aan eerste - en tweedejaarsstudenten verpleegkunde.
%	\end{cvitems}
%}

\cventry
{Docent} % Job title
{Thomas More hogeschool} % Organization
{Antwerpsestraat 97, Lier.} % Location
{2017 - heden} % Date(s)
{ % Description(s) of tasks/responsibilities
\begin{cvitems}
\item Docent van het wetenschappelijk vak "Evidentie en Onderzoek", 120-tal studenten.
\item Begeleider van BANABA studenten pediatrie bij hun bachelorproef.
\item Stagebegeleiding studenten verpleegkunde.
\item Evalueren van leerprocessen, bijsturen en actief werken rond de verdere uitbouw van de opleiding (stuurgroepen).
\item Geven van vaardigheidtrainingen (skillslab) aan eerste - en tweedejaarsstudenten verpleegkunde.
\end{cvitems}
}

%------------------------------------------------

\cventry
{Stagebegeleider} % Job title
{UCLL hogeschool} % Organization
{Herenstraat 49, Leuven.} % Location
{2017 - 2017} % Date(s)
{ % Description(s) of tasks/responsibilities
	\begin{cvitems}
	\item Stagebegeleiding van een vijftiental studenten verpleeg - en vroedkunde.
	\end{cvitems}
}

%------------------------------------------------

\cventry
{Leerkracht volwassenonderwijs.} % Job title
{CVO-VTI} % Organization
{Dekenstraat 3, 3000 Leuven.} % Location
{2015 - 2017} % Date(s)
{ % Description(s) of tasks/responsibilities
	\begin{cvitems}
		\item Docent "omgaan met dementie", "totaalzorg", "zorg voor woon - en leefklimaat", "hef - en tiltechnieken", "EHBO".
		\item Begeleiden van studenten bij hun eindopdracht.
		\item Deze opleiding bestaat uit  drie specifieke trajecten: logistiek medewerker, verzorgende en zorgkundige. Daarnaast verzorg ik de stagebegeleiding en was ik eindverantwoordelijke voor het vakoverleg van \textit{zorg voor woon - en leefklimaat} en het vak \textit{context van de zorgvragen}. 
		\item Ik was ook verantwoordelijk voor de graduele omschakeling van deze vakken, taken en examens naar de nieuwe onderwijsstandaard van permanente evaluatie. 
		In mijn lessen maakte ik zoveel mogelijk gebruik van de laatste nieuwe digitale ontwikkelingen en leermateriaal zoals digitaal stemmen, gebruik van apps, online quizzen, Powerpoint en multi-media.
	\end{cvitems}
}

%------------------------------------------------

\cventry
{Diabeteseducator} % Job title
{UZ Gasthuisberg} % Organization
{Herestraat 49, Leuven.} % Location
{2010 - 2015} % Date(s)
{ % Description(s) of tasks/responsibilities
	\begin{cvitems}
		\item Begeleiden, informeren en bijstaan van diabetespatiënten en familie gedurende de hele loopbaan van de ziekte. Afhankelijk van de zorgvraag en de noden van de pati\"{e}nt werd er aangepaste educatie voorzien.
		\item Het werk omvat ook een administratief luik: zorgen dat de conventies en zorgtrajecten in orde zijn, verslagen voor andere zorgverleners opmaken, vergaderingen bijwonen. % ...Ook dient men autonoom te kunnen werken in een team van 15 diabeteseducatoren. 
		\item In samenwerking met het multidisciplinair team had ik een relatief grote invloed in het uitwerken van het beleid van de afdeling.
		\item Ik was specifiek verantwoordelijk voor een aantal speficieke domeinen, i.e: implantpompen, endocriene tetsten, betaceltransplantatie, screening van familieleden.
	\end{cvitems}
}

%------------------------------------------------

%\cventry
%{Adjunct hoofdverpleegkundige.} % Job title
%{Rode Kruis, Leuven} % Organization
%{Dekenstraat 3, 3000 Leuven.} % Location
%{2010 - 2017} % Date(s)
%{ % Description(s) of tasks/responsibilities
%	\begin{cvitems}
%		\item Ik stond - in samenspraak met en in afwezigheid van de hoofdverpleegkundige- in voor de aansturing van de medewerkers en de creatie van een aangenaam werkklimaat
%		\item Daarnaast ondersteunde en adviseerde ik de hoofdverpleegkundige inzake planning en werkverdeling.
%		\item Ook werd ik ingeschakeld in de verpleegequipe en stond hierbij in voor het plannen, uitvoeren, coördineren en evalueren van de verpleegkundige zorgen.
%		\item Vanuit mijn affiniteit met het werkveld werkte ik (kwaliteits)projecten uit en adviseerde ik de hoofdverpleegkundige rond de optimalisatie van de dienstverlening.
%	\end{cvitems}
%}

%------------------------------------------------

\cventry
{Verpleegkundige} % Job title
{UZ Gasthuisberg, Leuven} % Organization
{Herestraat 49, Leuven.} % Location
{2006 - 2010} % Date(s)
{ % Description(s) of tasks/responsibilities
	\begin{cvitems}
		\item Verpleegkundige reuma - en endocrinologie.
	\end{cvitems}
}

%------------------------------------------------

\end{cventries}
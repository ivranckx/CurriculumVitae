\documentclass[10pt]{report}
\usepackage{a4,isolatin1, amssymb}
%mgcv.sty    markgittins@ukonline.co.uk
\newcommand{\aquatre}{\setlength{\oddsidemargin}{-1cm}
  \setlength{\evensidemargin}{0cm}\setlength{\textwidth}{17cm}
  \setlength{\topmargin}{-8mm}\setlength{\textheight}{24cm}
%  \setlength{\headheight}{0cm}\setlength{\headsep}{0cm}
  \setlength{\parindent}{0cm}
}

\newcommand{\name}[1]{{\LARGE \center \bf #1\\\markright{#1}}}

%\renewcommand{\today}{
%  \ifcase\month\or
%      January\or February\or March\or April\or May\or June\or
%      July\or August\or September\or October\or November\or December\fi,
%      \space\number\year}

\newcommand{\address}[1]{{\center #1}}
\newcommand{\hometel}[1]{{\tiny{(home)}} #1\\}
\newcommand{\worktel}[1]{{\tiny{(work)}} #1\\}
\newcommand{\mail}[1]{{\it #1}\\}
\newcommand{\url}[1]{{\it #1}\\}
\newcommand{\hligne}{\begin{tabular}{p{\textwidth}}\hline\ \\\end{tabular}}

\newcommand{\parag}[1]{\section*{\bf #1}\hligne}
\usepackage[safe]{textcomp}

\aquatre

\begin{document}
\pagestyle{headings}
\thispagestyle{empty}

\name{Curriculum Vitae Vranckx Ewout}
\ \newline

%\address{Town\\
%South Yorkshire\\
%U.K.\\}



%\begin{flushright}
%\hometel{0000 000000}
%\worktel{1010 1010101}

%\medskip

%\mail{email@host.co.uk}
%\url{http://www.host.co.uk}
%\end{flushright}

\parag{coordonn\'{e}es personelle}
		
\begin{tabular}[t]{ll}

Nom & Vranckx\\[1ex]
Pr\'{e}nom & Ewout\\[1ex]
Adresse & 118 Blvd. Anspach, 1000 Bruxelles, Belgique\\[1ex]
& Hazendreef 23, 2920 Heide kalmthout, Belgique\\[1ex]
Mobile & +32 (0)476 52 24 34 (GSM)\\[1ex]
E-mail & vranckxewout@hotmail.com\\[1ex]
%URL & http://joa.studentenweb.org\\[1ex]
Date de naissance & 14 septembre 1982, Merksem (Anvers), Belgique\\[1ex]
Nationalit\'{e} & Belge\\[1ex]

\end{tabular}
\ \\

\parag{\'{e}tudes}
\begin{itemize}

\item Specialisation cuisine gastronomique

  \begin{itemize}
  \item Lyc\'{e}e d'hotellerie PIVA Anvers  \hfill 01/09/2001 - 30/06/2003
    \begin{itemize}
  	\item \'{e}tude: cuisine
    \end{itemize}
      \end{itemize}
      


  \begin{itemize}
  \item Lyc\'{e}e d'hotellerie PIVA Anvers  \hfill 01/09/1998 - 30/06/2001
    \begin{itemize}
  	\item \'{e}tude: cuisine
    \end{itemize}    
  \end{itemize}

\item 
  \begin{itemize}
  \item Lyc\'{e}e d'hotellerie V.T.I Spijker, Hoogstraten \hfill 01/09/1994 - 30/06/1998
    \begin{itemize}
  	\item \'{e}tude: cuisine
    \end{itemize}    
  \end{itemize}
\end{itemize}
%\newpage

\parag{Stages}
\begin{itemize}

\item \textbf{\emph{Le Bernardin}} - New York City, Midtown Manhatten (USA)  \hfill 02/08/2007 - 15/08/2007
\begin{description}
 	\item Chef: Chris Muller \& Eric Ripert
	\item Michelin: \textborn \textborn \textborn
	\item stage int\'{e}resse
 \end{description}



\item \textbf{\emph{`ADPA` Alain Ducasse au Plaza Athen\'{e}e}} - Paris 8arr, France  \hfill 08/05/2005 - 18/05/2005
\begin{description}
 	\item Chef: Christophe Moret \& Alain Ducasse
	\item Michelin: \textborn \textborn \textborn
	\item stage pour la groupe Ducasse
 \end{description}

\item \textbf{\emph{`Relais \& chateaux`, Le Divellec}} - Paris 8arr, France  \hfill 14/09/2003 - 16/02/2004
\begin{description}
 	\item Chef: Jacques Le Divellec
	\item Michelin: \textborn \textborn
	\item Stage pour project Europ\'{e} 'Da Vinci'
 \end{description}

\item \textbf{\emph{Hotel du Palais}}, Biarritz, France  \hfill 05/04/2003 - 16/05/2003
\begin{description}
 	\item Chef: Jean-Marie Gautier
	\item Michelin: \textborn
	\item Stage \'{e}cole
 \end{description}

\item \textbf{\emph{'Relais \& Chateaux' Cordeillan-Bages}}, Pauillac (Medoc) France  \hfill 02/02/2002 - 15/04/2002
\begin{description}
 	\item Chef: Thierry Marx
	\item Michelin: \textborn \textborn
	\item Stage \'{e}cole
 \end{description}
\end{itemize}


%Carriere ---------------------------------------------
%\newpage
\parag{ Carri\`{e}re professionelle}

\begin{itemize}

\item \textbf{\emph{Comme Chez Soi}} - Bruxelles (Belgique)  \hfill 25/09/2007 - ...
\begin{description}
 	\item Chef: Lionel Rigolet \& Pierre Wynants
	\item Michelin: \textborn \textborn
	\item Chef de Partie
 \end{description}

\item \textbf{\emph{Pavillion `Le Doyen`}} - Paris 8arr, France  \hfill 01/02/2006 - 28/07/2007
\begin{description}
 	\item Chef: Christian Le Squer
	\item Michelin: \textborn \textborn \textborn 
	\item Demi-Chef de partie
 \end{description}


\item \textbf{\emph{Le Louis XV}} - Alain Ducasse - Monte-Carlo (Monaco)  \hfill 10/07/2005 -  30/12/2005
\begin{description}
 	\item Chef: Franck Cerutti \& Alain Ducasse
	\item Michelin: \textborn \textborn \textborn 
	\item Commis de cuisine
 \end{description}

\item \textbf{\emph{Spoon Food \& Wine}}, Alain Ducasse - Paris 8arr. (France)  \hfill 01/04/2004 - 08/07/2005
\begin{description}
 	\item Chef: Christophe Moret, David Bellin, St\'{e}phane Col\'{e}
	\item Chef de Partie
 \end{description}
\end{itemize}

\parag{Langues}

\begin{itemize}
\item \textbf{langues}
	\begin{itemize}
	\item N\'{e}erlandais: langue maternelle
	\item Anglais: parler et \'{e}crire parfaitement
	\item Francais: parler et \'{e}crire parfaitement
	\item Allemand: base
	\end{itemize}
	
\end{itemize}


 \par\nopagebreak\vfill\hfill \today

\end{document}

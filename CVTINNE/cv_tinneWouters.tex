%%%%%%%%%%%%%%%%%%%%%%%%%%%%%%%%%%%%%%%%%
% "ModernCV" CV and Cover Letter
% LaTeX Template
% Version 1.1 (9/12/12)
%
% This template has been downloaded from:
% http://www.LaTeXTemplates.com
%
% Original author:
% Xavier Danaux (xdanaux@gmail.com)
%
% License:
% CC BY-NC-SA 3.0 (http://creativecommons.org/licenses/by-nc-sa/3.0/)
%
% Important note:
% This template requires the moderncv.cls and .sty files to be in the same 
% directory as this .tex file. These files provide the resume style and themes 
% used for structuring the document.
%
%%%%%%%%%%%%%%%%%%%%%%%%%%%%%%%%%%%%%%%%%

%----------------------------------------------------------------------------------------
%	PACKAGES AND OTHER DOCUMENT CONFIGURATIONS
%----------------------------------------------------------------------------------------

\listfiles
\documentclass[11pt,a4paper,sans]{moderncv} % Font sizes: 10, 11, or 12; paper sizes: a4paper, letterpaper, a5paper, legalpaper, executivepaper or landscape; font families: sans or roman
\usepackage[dutch]{babel}
\usepackage{lmodern}
\usepackage[utf8]{inputenc}

\moderncvstyle{casual} % CV theme - options include: 'casual' (default), 'classic', 'oldstyle' and 'banking'
\moderncvcolor{orange} % CV color - options include: 'blue' (default), 'orange', 'green', 'red', 'purple', 'grey' and 'black'

\usepackage{lipsum} % Used for inserting dummy 'Lorem ipsum' text into the template

\usepackage[scale=0.75]{geometry} % Reduce document margins
%\setlength{\hintscolumnwidth}{3cm} % Uncomment to change the width of the dates column
%\setlength{\makecvtitlenamewidth}{10cm} % For the 'classic' style, uncomment to adjust the width of the space allocated to your name

%----------------------------------------------------------------------------------------
%	NAME AND CONTACT INFORMATION SECTION
%----------------------------------------------------------------------------------------

\firstname{Tinne} % Your first name
\familyname{Wouters} % Your last name
\title{\small{Master in de verpleeg - en vroedkunde.}}
% All information in this block is optional, comment out any lines you don't need
%\title{Curriculum Vitae}
\address{Oudenbos, 81}{3202, Rillaar, Aarschot}
\mobile{+0494134522}
\phone{+016438329}
%\fax{(000) 111 1113}
\email{woutetin@hotmail.com}
%\homepage{users.telenet.be/hosti}{users.telenet.be/hosti} % The first argument is the url for the clickable link, the second argument is the url displayed in the template - this allows special characters to be displayed such as the tilde in this example
%\extrainfo{additional information}
%\photo[70pt][0.4pt]{pictures/picture} % The first bracket is the picture height, the second is the thickness of the frame around the picture (0pt for no frame)
%\quote{Any intelligent fool can make things bigger, more complex, and more violent. It takes a touch of genius -- and a lot of courage -- to move in the %opposite direction.}% also optional

%----------------------------------------------------------------------------------------

\begin{document}
	\makecvtitle % Print the CV title
	\section{\textsc{Objectief}}
	\cvitem{}{Een boeiende en uitdagende job waarbij ik mensen kan begeleiden, adviseren en helpen - van welke aard hun problematiek ook mogen zijn. Daarnaast draag ik een open communicatie hoog in het vaandel, evenals occasionele bijscholingmogelijkheden en een vlotte collegiale samenwerking. Dit alles in samenwerking met een werkgever die verantwoordelijkheid en vertrouwen geeft.}
	\closesection{}
	
	\section{\textsc{Persoonlijke informatie}}
	\cvitem{Geboortedatum}{Maart 23, 1983}
	\cvitem{Geboorteplaats}{Geel, Belgi\"{e}}
	\cvitem{Nationaliteit}{Belg}
	\cvitem{Burgerlijke staat}{Samenwonend, moeder van Ilke, Arnout en Nand Vranckx.}
	\closesection{}
	
	\section{\textsc{Profiel}}
	\cvlistitem{Een gemotiveerde en gepassioneerde teamplayer}
	\cvlistitem{Zeer divers en ruim interessegebied, gaande van diabetes tot tropische systeemziekten}
	\cvlistitem{Steeds op zoek naar kennis om mezelf verder te ontwikkelen.}
	\cvlistitem{Beschik over capaciteiten om moeilijke materie verstaanbaar over te brengen.}
	\cvlistitem{Verantwoordelijkheidszin, ondernemend en vlot in omgang.}
	\closesection{}
	
	\section{\textsc{Talen}}
	\cvitem{Nederlands}{Moedertaal}
	\cvitem{Engels}{Zeer goede kennis.}
	\cvitem{Frans}{Aanvaardbare kennis.}
	\closesection{}
	
	\section{\textsc{Algemene interesses}}
	\cvitem{}{Tijdens mijn vrije tijd probeer ik zoveel mogelijk te genieten van mijn kindjes en partner. In de resterende tijd kan het lezen van een boek, wandelen of op stap gaan met vriendinnen mij ontspannen. }
	\closesection{}
	
	
	%----------------------------------------------------------------------------------------
	%	EDUCATION SECTION
	%----------------------------------------------------------------------------------------
	\section{\textsc{Opleiding}}
	
	\cventry{\textbf{2015-heden}}{Specifieke lerarenopleiding}{}{\newline CVO-VTI}{Leuven, Belgi\"{e}}}
	\cvitem{}{}
	
	\cventry{\textbf{2010-2011}}{Postgraduaat diabeteseducator}{}{\newline Artevelde hogeschool}{Gent, Belgi\"{e}}{}
	\cvitem{}{}
	
	\cventry{\textbf{2009-2009}}{Postgraduaat Tropische geneeskunde voor verpleeg- en vroedkundigen}{}{\newline Instituut voor Tropische Geneeskunde}{Antwerpen, Belgi\"{e}\htmladdnormallink{ $^{*}$}{http://www.itg.be/itg/}}{}
	\cvitem{}{}
	
	\cventry{\textbf{2004-2006}}{Master in de Verpleegkunde en vroedkunde}{\newline Departement ziekenhuiswetenschappen}{Katholieke Universiteit Leuven}{Belgi\"{e}\htmladdnormallink{ $^{*}$}{http://www.kuleuven.be/}}{}
	\cvitem{Thesis}{\emph{De betrokkenheid en de beleving van de vroedvrouw bij prenatale diagnostiek en zwangerschapsafbreking: een kwalitatieve studie.}}
	\cvitem{Supervisors}{Prof. C. Gastmans, Prof. B. Dierckx de Casterl\'{e}}
	\cvitem{}{}
	
	\cventry{\textbf{2001-2004}}{Graduaat in de Vroedkunde}{\newline Departement Gezondheidszorg}{Katholieke Hogeschool Kempen}{Lier, Belgi\"{e}\htmladdnormallink{ $^{*}$}{http://www.khk.be/khk04/}}{}
	\cvitem{Thesis}{\emph{Medische en alternatieve behandeling van reflux.}}
	\cvitem{Supervisors}{Mevr. K. Grauwels}
	\cvitem{}{}
	\closesection{}
	%----------------------------------------------------------------------------------------
	%	WORK EXPERIENCE SECTION
	%----------------------------------------------------------------------------------------
	
	\section{Werkervaring}
	%\cvitem{}{}
	%\subsection{\textbf{2006-2009}}{Professionele ervaring}{}{}{}{}
	
	\cvitem{\small 2017-2018}{\textbf{docent }  Thomas More Lier \newline \newline \small Lesgeven en begeleiden van studenten verpleegkunde. Daarnaast verzorg ik de stagebegeleiding en was ik verantwoordelijke voor het vak Evidentie en onderzoek. \newline Ik was ook verantwoordelijk voor de graduele omschakeling van deze vakken, taken en examens naar de nieuwe onderwijsstandaard . In mijn lessen maakte ik zoveel mogelijk gebruik van de laatste nieuwe digitale ontwikkelingen en leermateriaal zoals digitaal stemmen, gebruik van apps, online quizzen, Powerpoint en multi-media.}
	
	\cvitem{}{{Leerkracht volwassenonderwijs}  CVO-VTI, Leuven \newline \newline \small Lesgeven en begeleiden van studenten zorgkunde. Deze opleiding bestaat uit  drie specifieke trajecten: logistiek medewerker, verzorgende en zorgkundige. Daarnaast verzorg ik de stagebegeleiding en was ik eindverantwoordelijke voor het vakoverleg van \textit{zorg voor woon - en leefklimaat} en het vak \textit{context van de zorgvragen}. \newline Ik was ook verantwoordelijk voor de graduele omschakeling van deze vakken, taken en examens naar de nieuwe onderwijsstandaard van permanente evaluatie. In mijn lessen maakte ik zoveel mogelijk gebruik van de laatste nieuwe digitale ontwikkelingen en leermateriaal zoals digitaal stemmen, gebruik van apps, online quizzen, Powerpoint en multi-media.}}
	
	\cvitem{\small 2010-2015}{\textbf{Diabeteseducator}  UZ Gasthuisberg, Leuven \newline \newline \small Begeleiden, informeren en bijstaan van diabetespati\"{e}nten en familie gedurende de hele loopbaan van de ziekte. Afhankelijk van de zorgvraag en de noden van de pati\"{e}nt zal er aangepaste begeleiding en educatie voorzien worden.}
	\cvitem{}{\small Mijn huidig werk omvat ook een administratief luik: zorgen dat de conventies en zorgtrajecten in orde zijn, verslagen voor andere zorgverleners opmaken, vergaderingen bijwonen, ...Ook dient men autonoom te kunnen werken in een team van 15 diabeteseducatoren. }
	\cvitem{}{\small In samenwerking met het multidisciplinair team heb ik een relatief grote invloed in het uitwerken van het beleid van de afdeling.}
	\cvitem{}{\small Ik bent specifiek verantwoordelijk voor een aantal speficieke domeinen - gerelateerd aan diabetes: implantpompen, endocriene tetsten, betaceltransplantatie, screening van familieleden.}  
	
	\cvitem{}{}
	\cvitem{\small 2009-2009}{\textbf{Adjunct hoofdverpleegkundige}   Rode Kruis, Leuven}
	\cvitem{}{\small Ik stond - in samenspraak met en in afwezigheid van de hoofdverpleegkundige- in voor de aansturing van de medewerkers en de creatie van een aangenaam werkklimaat}
	\cvitem{}{\small Daarnaats ondersteunde en adviseerde ik de hoofdverpleegkundige inzake planning en werkverdeling.}
	\cvitem{}{\small Ook werd ik ingeschakeld in de verpleegequipe en stond hierbij in voor het plannen, uitvoeren, co\"{o}rdineren en evalueren van de verpleegkundige zorgen.}
	\cvitem{}{\small Vanuit mijn affiniteit met het werkveld werkte ik (kwaliteits)projecten uit en adviseerde ik de hoofdverpleegkundige rond de optimalisatie van de dienstverlening. }  
	\cvitem{}{}
	
	\cvitem{\small 2007-2009}{\textbf{Verpleegkundige reuamtologie en endocrinologie} UZ gasthuisberg, Leuven}
	\cvitem{}{}
	\cvitem{\small 2007-2007}{\textbf{Vroedvrouw}, UZ Gasthuisberg, materniteit}
	\cvitem{}{}
	\cvitem{\small 2006-2007}{\textbf{Verpleegkundige reumatologie en endocrinolgie}, UZ Gasthuisberg, Leuven}
	\cvitem{}{}
	\closesection{}
	
	%----------------------------------------------------------------------------------------
	%	COVER LETTER
	%----------------------------------------------------------------------------------------
	
	% To remove the cover letter, comment out this entire block
	
	%\clearpage
	%\recipient{KHLeuven, Departement Gezondheidszorg en Technologie}{Herestraat 49\\3000, Leuven} % Letter recipient
	%\date{\today} % Letter date
	%\opening{Geachte Heer, Mevrouw} % Opening greeting
	%\closing{Met vriendelijke groeten,} % Closing phrase
	%\enclosure[Attached]{curriculum vit\ae{}} % List of enclosed documents
	%
	%\makelettertitle % Print letter title
	%
	%Sollicitatie lector voor de basisopleiding Verpleegkunde\newline
	%
	%Op jullie website heb ik gelezen dat u op zoek bent naar een \textit{Praktijklector Verpleegkunde} te Leuven. De jobomschrijving die ik daarin terugvond, spreekt me heel erg aan.\newline\newline
	%Als afgestudeerde master in de verpleegkunde en vroedkunde werk ik momenteel als diabeteseducator in het Universitair ziekenhuis Gasthuisberg. Door het uitoefenen van deze job ben ik vertrouwd geraakt met het afstemmen van mijn kennis op de noden van de pati\"{e}nt; Daarnaast beschik ik over een goede actuele kennis, de nodige verantwoordelijkheidszin en de vereiste vaardigheden met betrekking tot de verpleegkundige zorg.\newline
	%Na meerdere jaren actief te zijn in de gezondheidszorg ben ik op zoek naar een nieuwe uitdaging. De participatie in de opleiding van de nieuwe generatie verpleegkundigen zou me enorm motiveren. \newline
	%Als bijlage vindt u mijn cv terug. Deze kom ik graag toelichten tijdens een persoonlijk gesprek.\newline\newline
	%
	%
	%\makeletterclosing % Print letter signature
	
	%----------------------------------------------------------------------------------------
	
\end{document}